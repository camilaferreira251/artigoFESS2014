\section{Introdução}
\label{sec:introducao}

O governo federal brasileiro vem nos últimos anos buscando melhorias nos
seus processos de desenvolvimento e adoção de software.
%
Desde 2003, a recomendação da adoção de software livre passou a ser uma política
incentivada na esfera federal, inicialmente com a criação do
Guia Livre\footnote{governoeletronico.gov.br/acoes-e-projetos/guia-livre}.
%
Hoje, tal iniciativa é coordenada pelo Comitê Técnico de Implementação de
Software Livre do Governo Federal\footnote{softwarelivre.gov.br}.


No contexto da promoção do software livre no governo federal, a
Secretaria de Logística e Tecnologia da Informação (SLTI) do Ministério do
Planejamento, Orçamento e Gestão (MP) inaugurou em 2007, o Portal do Software
Público Brasileiro (SPB), que é um sistema web que se consolidou como
um ambiente de compartilhamento de projetos de software no governo.
%
Por exemplo, a Instrução Normativa
04/2012\footnote{governoeletronico.gov.br/biblioteca/arquivos/instrucao-normativa-no-04-de-12-de-novembro-de-2010}
indica que os gestores devem consultar as soluções existentes no Portal do SPB
antes de realizar uma contratação de software.
%
Hoje, com o portal do SPB tem cerca de 69 comunidades de
desenvolvimento e mais de 200.000 usuários cadastrados.

Entretanto, a evolução do SPB foi comprometida, desde 2009, quando a plataforma
do SPB não acompanhou a evolução do seu \textit{framework} base,
o \emph{OpenACS}\footnote{openacs.org}.
%
Com isso, não tendo versões lançadas a partir daquele ano.

Nesse contexto, um dos passos para a concretização de uma nova plataforma para o
SPB é a integração com novas tecnologias, desde uma plataforma colaborativa até sistemas
de controle de versão e de monitoramento da qualidade do código-fonte.
%
Para concretizar a evolução do SPB, a Universidade de Brasília está coordenando
tal processo, através de uma equipe heterogênea de alunos, professores e
profissionais, que estão aplicando práticas colaborativas de desenvolvimento
de software e gestão de projeto que impactam no projeto e, em especial,
na aprendizagem dos envolvidos.



