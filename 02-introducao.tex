\section{Introdução}
\label{sec:introducao}

O Portal do Software Público Brasileiro (SPB) é um projeto da Secretaria de Logística e Tecnologia da Informação - SLTI do Ministério do Planejamento, Orçamento e Gestão - MP. O Portal foi inaugurado em 2007, e na prática,
é um sistema web que se consolidou como um ambiente de compartilhamento de
de softwares. 
%
%Oferece um espaço (comunidade) para cada software. A comunidade é composta por
%fórum, notícias, {\it chat}, armazenamento de arquivos, {\it downloads}, {\it wiki}, lista de
%prestadores de serviços, usuários, coordenadores, entre outros recursos.
%
Teve um crescimento expressivo contando, hoje, com cerca de 69 comunidades de
desenvolvimento e mais de 200.000 usuários cadastrados.
%
O SPB abrange alguns grupos de interesse cujo objetivo é agrupar usuários em torno de um determinado tema ligados ao SPB. O 4CMBr\footnote{http://softwarepublico.gov.br/4cmbr/xowiki/Principal}, é um grupo de interesse voltado para soluções
de tecnologia para municípios, o 5CQualiBr\footnote{http://softwarepublico.gov.br/5cqualibr/xowiki/}, é um grupo que trabalha para
evoluir a qualidade do SPB, e o 4CTECBr\footnote{http://www.softwarepublico.gov.br/4ctecbr/wiki/Principal} visa discutir sobre as tecnologias livres.
%
Além dos grupos de interesse o ecossitema abrange o ambiente Mercado Público Virtual\footnote{http://www.softwarepublico.gov.br/mpv/}, que é um
grupo de empresas e pessoas que prestam serviços nos softwares ofertados no
Portal, e o AvaliaSPB\footnote{http://www.softwarepublico.gov.br/5cqualibr/avaliaspb/wiki/principal}, que é responsável pelo processo de disponibilização dos softwares candidatos à software
público.
%
%O Portal do SPB foi criado para atender as demandas deste grande ecossistema, são diversos ambientes que precisam ser administrados. A plataforma escolhida na ocasião da
%criação foi o framework OpenACS, que continua sendo utilizada na versão atual.

A evolução do SPB foi comprometida desde 2009, quando framework OpenACS foi 
descontinuado. Com isso, não tendo versões lançadas a partir daquele ano.
%
Este evento contribuiu para que o portal passasse por um processo de desatualização tecnológica, ou seja qualquer evolução a ser realizada na ferramenta demandava muito tempo e esforço dos analistas. Sendo assim o portal atual conta com uma série de problemas, especialmente a ausência de uma ferramenta para desenvolvimento colaborativo dos softwares e a falta de integração com ambientes colaborativos
externos, as redes sociais, por exemplo.
Por isso, hoje, é necessária a evolução para novas tecnologias que tenham maior
suporte das comunidades de desenvolvimento, que utilize linguagens de programação
com maior rapidez de aprendizagem e de desenvolvimento e que permita a integração
com ambientes colaborativos externos.
%
Além disso, é preciso realizar a manutenção evolutiva das funcionalidades
existentes e também o desenvolvimento de novas funcionalidades para o Portal
do SPB.
%

Um dos passos para a concretização de uma nova geração do Portal SPB é a
integração com novas tecnologias, desde uma plataforma colaborativa até sistemas
de controle de versão e de monitoramento da qualidade do código fonte,
gerenciadas e apresentadas em uma plataforma integrada no {\it back-end} e, em especial,
no {\it front-end} para que os usuários e as comunidades tenham um conjunto de
recursos para encontrarem os projetos, bem como colaborarem em torno de um
sofware público.
%
Mesmo com as limitações citadas, o Portal do Software Público Brasileiro teve
em 2013 mais de 600 mil visitantes únicos, com mais de 1 milhão de acessos,
gerando mais de 16 milhões de visitadas nas páginas, com um total de mais de 49
milhões de hits \footnote{O número de vezes que os visitantes visualizaram uma
determinada página no site.} no Portal SPB.
%
Avaliando apenas as comunidades dos projetos I3Geo\footnote{http:\/\/www.softwarepublico.gov.br\/ver-comunidade?community\_id=1444332}, CAU\footnote{http:\/\/www.softwarepublico.gov.br\/ver-comunidade?community\_id=48535178}, CACIC\footnote{http:\/\/www.softwarepublico.gov.br\/ver-comunidade?community\_id=3585} e Geplanes\footnote{http:\/\/www.softwarepublico.gov.br\/ver-comunidade?community\_id=20483099}, houveram
mais de 15 mil downloads e 4 mil mensagens trocadas nos fóruns.
%
Essa amostra ilustra bem o potencial do Portal do Software Público Brasileiro que mesmo com alguns problemas possui uma grande quantidade de acessos, softwares e usuários. A intenção é aumentar as expectativas dos usuários e colaboradores com a evolução do Portal e
do modelo em si. Para materializar essa evolução o MP firmou um Acordo de Cooperação Técnica com a UNB visando que alunos e professores possam apoiar na criação do Novo Portal do Software Público Brasileiro. 

%TODO: linkar com UnB, projeto de pesquisa, educação etc
%TODO: organização do artigo

