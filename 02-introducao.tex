\section{Introdução}
\label{sec:introducao}

O governo federal brasileiro vem nos últimos anos buscando melhorias nos
seus processos de desenvolvimento e adoção de software.
%
Desde 2003, a recomendação da adoção de software livre passou a ser uma política
incentivada na esfera federal, inicialmente com a criação do
Guia Livre\footnote{governoeletronico.gov.br/acoes-e-projetos/guia-livre}.
%
Hoje, tal iniciativa é coordenada pelo Comitê Técnico de Implementação de
Software Livre do Governo Federal\footnote{softwarelivre.gov.br}.


No contexto da promoção do software livre no governo federal, a
Secretaria de Logística e Tecnologia da Informação (SLTI) do Ministério do
Planejamento, Orçamento e Gestão (MP) inaugurou em 2007, o Portal do Software
Público Brasileiro (SPB), que, na prática, é um sistema web que se consolidou como
um ambiente de compartilhamento de projetos de software no governo.
%
Por exemplo, a Instrução Normativa
04/2012\footnote{governoeletronico.gov.br/biblioteca/arquivos/instrucao-normativa-no-04-de-12-de-novembro-de-2010}
indica que os gestores devem consultar as soluções existentes no Portal do SPB
antes de realizar uma contratação de software.
%
Hoje, com o portal do SPB tem cerca de 69 comunidades de
desenvolvimento e mais de 200.000 usuários cadastrados.

Entretanto, a evolução do SPB foi comprometida, desde 2009, quando a plataforma
do SPB não acompanhou a evolução do seu arcabouço base,
o \emph{OpenACS}\footnote{openacs.org}.
%
Com isso, não tendo versões lançadas a partir daquele ano.
%
Dessa forma, o SPB passou um processo de estagnação tecnológica e qualquer
evolução a ser realizada na ferramenta demandava muito tempo e esforço dos
analistas da SLTI/MP.
%
Sendo assim, o SPB atual conta com uma série de problemas, especialmente a
ausência de uma ferramenta para desenvolvimento colaborativo e um ambiente
de rede social para as comunidades do projetos.

Nesse contexto, um dos passos para a concretização de uma nova plataforma para o
SPB é a integração com novas tecnologias, desde uma plataforma colaborativa até sistemas
de controle de versão e de monitoramento da qualidade do código fonte,
gerenciadas e apresentadas em uma plataforma integrada no {\it back-end} e, em especial,
no {\it front-end} para que os usuários e as comunidades tenham um conjunto de
recursos para encontrarem os projetos, bem como colaborarem em torno de um
sofware público.

Mesmo com as limitações citadas, o Portal do Software Público Brasileiro teve
em 2013 mais de 600 mil visitantes únicos, com mais de 1 milhão de acessos,
gerando mais de 16 milhões de visitas nas páginas, com um total de mais de 49
milhões de \emph{hits}. no Portal SPB.
%
Avaliando apenas os principais projetos, houveram mais de 15 mil downloads e
4 mil mensagens trocadas nos fóruns.
%
Essa amostra ilustra bem o potencial do SPB que mesmo com alguns problemas
possui uma boa quantidade de acessos, softwares e usuários.


Para concretizar a evolução do SPB, a Universidade de Brasília está coordenando
tal processo, através de uma equipe heterogêne de alunos, professores e
profissionais, que estão aplicando práticas colaborativas de desenvolvimento
de software e gestão de projeto que impactam no projeto em si e, em especial,
na aprendizagem dos alunos envolvidos, que estão tendo contato com métodos ágeis como 
XP e Scrum, além de novas ferramentas de suporte ao desenvolvimento de software. 
%
Portanto, neste artigo iremos relatar,
%
na Seção~\ref{sec:arquitetura}, apresentamos uma visão geral da arquitetura e
das tecnologias livres que estão sendo usadas e evoluídas para compor a nova
plataforma do SPB;
%
na Seção~\ref{sec:metodologia}, as
metodologias usadas e as principais práticas adaptadas à equipe em questão;
%
na Seção~\ref{sec:relato}, discutimos os resultados de uma pesquisa de
levantamento da aprendizagem dos alunos do projeto;
%
na Seção~\ref{sec:conclusao}, apresentamos as considerações finais deste relato
de experiência. 


