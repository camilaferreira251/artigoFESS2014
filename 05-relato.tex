\section{Relato de Experiência}
\label{sec:relato}

Para verificar o quanto o projeto está auxiliando a formação do aluno em diferentes
aspectos, foi aplicado um questionário na equipe de desenvolvimento do Novo Portal 
do Software Público, onde os mesmos poderiam expressar o quanto estão aprendendo 
com o projeto.

O questioário foi respondido por 17 integrantes do projeto que têm entre 20 e 26 anos, os quais então entre o 5º e o 10º semestre do curso de Engenharia de Software.

A primeira questão pretendia averiguar o nível de conhecimento dos entrevistados, em uma escala de 1 a 5, em relação às ferramentas que estão sendo utilizadas no projeto. Como resultados tivemos que 63\% responderam nível 3, 19\% responderam nível 2 e 19\% responderam nível 4.

A segunda questão pretendia averiguar o quanto o projeto está contribuindo para a formação do entrevistado. Em uma escala de 0 a 5, 69\% dos entrevistados responderam "5 -Contribui muito, mais que determinadas disciplinas", 19\% dos entrevistados responderam "4 - Contribui, no mesmo nível de muitas disciplinas" e 13\% dos entrevistados responderam "3 - Contribui, da mesma forma que um estágio convencional".

%\begin{figure}[htpb]
 % \begin{center}
  %  \includegraphics[width=.37\textwidth]{images/chart1.png}
  %\end{center}
  %\caption{Respostas do nível de contribuição do projeto na sua formação}
  %\label{fig:core_concurrent}
%\end{figure} 

A terceira questão está relacionada, também em uma escala de 0 a 5, a quanto o projeto atrapalha a sua performance nas disciplinas da graduação. Como resultados, 19\% dos entrevistados responderam "3 - Atrapalha minha graduação da mesma forma que um estágio", 50\% dos entrevistados responderam "2 - Atrapalha moderadamente", 19\% dos entrevistados responderam "1 - Atrapalha pouco, menos que um estágio" e 13\% dos entrevistados responderam "0  Não atrapalha em nada na minha graduação".

%\begin{figure}[htpb]
 % \begin{center}
  %  \includegraphics[width=.37\textwidth]{images/chart2.png}
  %\end{center}
  %\caption{Respostas do nível de performance nas disciplinas de graduação}
  %\label{fig:core_concurrent}
%\end{figure} 

A quarta questão averigua o quanto, em uma escala de 0 a 5, os conhecimentos adiquiridos durante a execução do projeto ajudam os entrevistados nas disciplinas da graduação. Como resultados, 7\% dos entrevistados responderam "5 - Ajuda muito, sempre utilizo os conhecimentos adiquiridos nas disciplinas da graduação	", 47\% respondeu "4  Ajuda muito, utilizo os conhecimentos com frequência nas disciplinas da graduação", 27\% respondeu "3 - Ajuda, utilizo os conhecimentos nas disciplinas da graduação" 13\% respondeu "2 - Ajuda moderadamente, as vezes utilizo os conhecimentos adquiridos no projeto" e 7\% respondeu "1 – Ajuda um pouco, utilizo pouco os conhecimentos do projeto".

%\begin{figure}[htpb]
 % \begin{center}
  %  \includegraphics[width=.37\textwidth]{images/chart3.png}
  %\end{center}
  %\caption{Respostas do nível de conhecimento adquirido}
  %\label{fig:core_concurrent}
%\end{figure} 

E a quinta questão perguntou o quanto, em uma escala de 0 a 5, o entrevistado acredita que o projeto do Novo SPB vai contribuir em experiência para o mercado de trabalho. Como resultados, 38\% responderam "5  O projeto me tornará experiente para o mercado de trabalho", 38\% responderam "4 - O projeto me dará muita experiência para o mercado de trabalho" e 25\% responderam "3 - O projeto me dará uma boa experiência para o mercado de trabalho".

%\begin{figure}[htpb]
 % \begin{center}
  %  \includegraphics[width=.37\textwidth]{images/chart4.png}
  %\end{center}
  %\caption{Respostas do nível de experiência para o mercado de trabalho}
  %\label{fig:core_concurrent}
%\end{figure} 

\subsection{Discussão dos resultados}

Já era esperado o resultado da questão relacionada ao nível de experiência da equipe com as ferramentas a serem utilizadas pois, era sabido que por serem alunos de graduação já poderiam ter tido alguma experiência com as ferramentas nas disciplinas sem chegar a ser usuário avançado ou desenvolvedores experientes nas mesmas, mas teriam algum conhecimento que auxiliaria na execução das atividades.

O projeto tem como objetivo auxiliar na formação dos alunos e por isso foi questionada a opinião dos alunos à respeito da contribuição trazida pelo projeto para sua formação, obtivemos respostas positivas para essa pergunta pois o projeto proporciona um ambiente de aprendizado e conhecimento de novas tecnologias e metodologias de desenvolvimento além de adiantar problemáticas que ainda serão vistas nas disciplinas da graduação.

Sabendo que o projeto retira do aluno horas semanais que poderiam estar destinadas ao estudo para as disciplinas da graduação, trouxemos a questão de quanto o projeto atrapalha no desempenho das disciplinas da graduação. As respostas foram mais variadas, alguns responderam que não atrapalha em nada na graduação, outros que atrapalham menos ou da mesma forma que um estágio e a maioria respondeu que atrapalha moderadamente, ou seja, o projeto está atrapalhado na graduação mas é compensado pelo conhecimento adquirido já que nenhum entrevistado respondeu que está se sentindo prejudicado por causa do projeto.

O projeto têm auxiliado nas disciplinas da graduação, segundo os entrevistados, como já mencionado, sendo assim o projeto adianta para os integrantes alguns assuntos que ainda seriam vistos nas disciplinas de graduação, diminuindo a curva de aprendizados dessas disciplinas durante a graduação.

Uma preparação para o mercado de trabalho é importante e segundo os entrevistados o projeto está trazendo uma boa experiência profissional para os seus integrantes. O contexto de um projeto real com prazos e necessidades específicas em diferentes áreas da engenharia de software devem ter sidos pensadas pelos entrevistados para a resposta desta questão. 

O projeto também tem beneficiado o aprendizado dos alunos em temas relacionados a Administração Pública pois o Portal do SPB é um sistema web que também visa beneficiar aos órgãos das esferas federal, estadual e municipal. Essa relação com o governo auxilia tanto no crescimento profissional, uma vez que estabelece um contato com usuários de governo, como uma visão sobre negócios que envolvem toda a Administração Pública.
