\section{Metodologia}
\label{sec:metodologia}

\subsection{A Equipe}
\label{sec:equipe}

A equipes técnica é formada por engenheiros de \textit{software} e artistas-designers, na sua maioria representado por alunos de gradução e mestrado da Universidade de Brasília, os quais estão tendo sua formação complementada por meio dessa oportunidade de participarem como protagonistas em um projeto de grande magnitude.
A equipe possui diferentes experiências e qualificações, adequadas a dinâmica e complexidade do projeto. Além de alunos, a equipe também é composta pelos professores coordenadores do projeto e por pesquisadores relacionados a comunidade de \textit{software} livre, os quais atuam tanto na execução quanto na coordenação das 
atividades técnicas não trivais do projeto.

Com isso, a transferência de tecnologias e do conhecimento acontece primeiro pela interação direta entre
os alunos bolsistas e pesquisadores do projeto e profissionais da Secretária de Logística e Tecnologia da Informação do Ministério do Planejamento (SLTI/MP), inclusive nas decisões de escolhas
técnicas, formas de processo, metodologias e etc. 

Além disso, a equipe técnica de cerca de vinte e quatro integrantes está dividida em subequipes que atuam, no momento, em três frentes: Colab, Noosfero e \textit{Design}. Em suma, a subequipe do Noosfero está desenvolvendo um plugin com novas funcionalidades para o Novo Portal do Software Público, a subequipe do Colab está trabalhando, no momento, com configurações e integração de ferramentas e a 
subequipe do \textit{Design} está trabalhando com a adaptação das interfaces das ferramentas integradas com o Colab.


\subsection{Métodos de desenvolvimento e Gestão do Projeto}
\label{sec:metodo-gestao}

A Engenharia de Software tem evoluído suas práticas e metodologias em busca de padrões que regem o desenvolvimento de software de qualidade dentro dos escopos, custos e prazos desejados. 
%
Dada a oportunidade de adoção de métodos ágeis no desenvolvimento do presente trabalho, a 
metodologia utilizada é baseada em uma combinação das metodologias Scrum e Extreme Programming. Destacam que XP e Scrum complementam um ao outro bem, com o XP provendo suporte para aspectos mais técnicos enquanto o Scrum provê práticas e técnicas para gerenciamento, planejamento e acompanhamento. Assim,com base nas experiências destacadas em~\cite{schwaber2001} e~\cite{fitzgerald2006}, na motivação de adoção de métodos ágeis no desenvolvimento de software moderno, serão apresentadas os métodos XP e Scrum, suas principais características e práticas que serão utilizadas no desenvolvimento do presente projeto.


\subsubsection{Ciclos de Produção}

O tempo de duração do projeto foi organizado em sete \textit{Releases}, cada uma com duração de quatro meses. Ao final de cada \textit{release} é realizada uma entrega. Ou seja, serão ao todo sete etapas de resultados a serem disponibilizadas ao longo da execução do projeto.
Para homologação e ateste técnico, é disponibilizado um resultado prévio um mês antes da entrega da \textit{Release}, chamadas de \textit{Releases} Candidatas. Assim, a partir da disponibilização do resultado da \textit{release} candidata, temos trinta dias para ajustes, correções e testes finais. 

Além disso, são disponibilizados resultados intermediários mensalmente. Trata-se da oportunidade da equipe da SLTI/MP poder fornecer \textit{feedbacks} de possíveis alterações ou novas funcionalidades, as quais são tratados como novos itens no \textit{backlog} para priorização.


\subsubsection{Programação em Par}

A equipe utiliza a prática de programação em par para o desenvolvimento de todas as histórias. Temos observado que tal prática facilita o nivelamento e a disseminação do conhecimento entre todos os membros da equipe sem atrapalhar a produtividade da mesma, pelo contrário, com essa prática a validação e a verificação da história é mais confiável por estar sendo feito por pelo menos duas pessoas.


\subsubsection{Comunicação da Equipe}

Devido a formação diversificada, a disponibilidade de horário variadas e a distribuição geográfica da equipe (com integrantes de Brasília, da Bahia e de São Paulo), precisamos definir algumas práticas que facilitassem a comunicação entre os integrantes. 

Para manter o controle das atividades do projeto, utilizamos a ferramenta de gerenciamento Redmine com configuração ágil, onde são mantidos o \textit{product backlog} com as histórias de usuários, as histórias técnicas e suas respectivas tarefas. Para cada história, há um responsável associado, o qual responde pelo progresso da história, e uma pontuação, que corresponde ao esforço planejado para a mesma.

Para informar o \textit{status} das tarefas do projeto e evitar falhas de comunicação mantemos uma lista de e-mail com todos os integrantes do projeto, na qual é relatado diariamente o \textit{status} das histórias da \textit{sprint} atual e eventuais problemas que precisam ser resolvidos. 


\subsubsection{Eventos}

Baseado nos eventos da metodologia ágil \textit{Scrum}, a equipe realiza algumas cerimônias, dentre elas:

\begin{itemize}
 \item \textit{Stand-up}: a equipe realiza \textit{stand-up} diários fisicamente, onde cada frente de trabalho comunica o que foi feito, o que está sendo feito e eventuais problemas. Para que os integrantes da equipe localizados em estados diferentes acompanhem o \textit{stand-up}, é utilizada uma ferramenta para \textit{hangout}.
 \item Reuniões de planejamento: no início de cada \textit{sprint}, é realizada uma reunião de planejamento na qual são definidas as histórias que seram desenvolvidades e essas histórias são pontuadas com a prática \textit{Planning Poker}.
 \item Reuniões de encerramento: ao final de cada \textit{sprint}, é realizada uma reunião de encerramento, onde é apresentado o que foi desenvolvido e concluído para a \textit{sprint} e o que pode ser melhorado.
\end{itemize}


