\section{Metodologia}
\label{sec:metodologia}

%A equipe é formada, em sua maioria, por estudantes, 16 de graduação e 5 de
%pós-graduação, de engenharia de \textit{software}, ciência da computação, artes
%e desenho industrial, da Universidade de Brasília e da Universidade de São
%Paulo, os quais estão tendo sua formação complementada por meio dessa
%oportunidade de participarem como protagonistas em um projeto de grande
%magnitude, o qual envolve dezenas de comunidades e milhares de usuários.
%
%Cinco professores coordenam as subequipes do projeto e dois desenvolvedores,
%criadores da duas principais ferramentadas livres adotadas (Colab e Noosfero), atuam como
%arquitetos de software, liderando as decisões técnicas.
%
%Além deles, três analistas da SLTI/MP acompanham e também participam da gestão
%do projeto e das decisões tecnológicas.
%
%Em resumo, os integrantes estão divididos em subequipes que atuam, no momento,
%em quatro frentes: \emph{Proxy} de Integração, Rede Social, Monitoramento de
%código-fonte e \textit{Design} de interação.

Pelo contexto colaborativo e empírico do desenvolvimento de software livre,
adotamos algumas práticas ágeis com base nas experiências destacadas em
~\cite{schwaber2001} e~\cite{fitzgerald2006}, sobre a motivação de adoção dos métodos ágeis, XP e Scrum, no desenvolvimento de software moderno, as quais são apresentadas a seguir.

%TODO{Descrever melhor a dinâmica do trabalho da equipe...}

%\subsubsection{Ciclos de Produção}

Para melhor gerenciamento a duração total do projeto foi dividida em sete \textit{Releases}, cada uma com duração de quatro meses. Ao final de cada \textit{release} é realizada uma entrega. Ou seja, serão ao todo sete etapas de resultados a serem disponibilizadas ao longo da execução do projeto.
Para homologação e ateste técnico, é disponibilizado um resultado prévio um mês antes da entrega da \textit{Release}, chamadas de \textit{Releases} Candidatas. Assim, a partir da disponibilização do resultado da \textit{release} candidata, temos trinta dias para ajustes, correções e testes finais. 

Além disso, são disponibilizados resultados intermediários mensalmente. Trata-se da oportunidade da equipe da SLTI/MP poder fornecer \textit{feedbacks} de possíveis alterações ou novas funcionalidades, as quais são tratados como novos itens no \textit{backlog} para priorização.

%\subsubsection{Programação em Par}

A equipe utiliza a prática de programação em par para o desenvolvimento de todas as histórias de usuários, que representam as funcionalidades a serem desenvolvidas, e histórias técnicas. Temos observado que tal prática facilita o nivelamento e a disseminação do conhecimento entre todos os membros da equipe sem atrapalhar a produtividade da mesma, pelo contrário, com essa prática a validação e a verificação da história é mais confiável por estar sendo feito por pelo menos duas pessoas.

%\subsubsection{Comunicação da Equipe}

Devido a formação diversificada, a disponibilidade de horário variadas e a distribuição geográfica da equipe (com integrantes de Brasília, da Bahia e de São Paulo), precisamos definir algumas práticas que facilitassem a comunicação entre os integrantes. 

Para informar o \textit{status} das tarefas do projeto e evitar falhas de comunicação mantemos uma lista de e-mail com todos os integrantes do projeto, na qual é relatado diariamente o \textit{status} das histórias da \textit{sprint} atual e eventuais problemas que precisam ser resolvidos. 

Para manter o controle das atividades do projeto, utilizamos a ferramenta de gerenciamento Redmine com configuração ágil, onde são mantidos o \textit{product backlog} com as histórias de usuários, as histórias técnicas e suas respectivas tarefas. Para cada história, há um responsável associado, o qual responde pelo progresso da história, e uma pontuação, que corresponde ao esforço planejado para a mesma.

%\subsubsection{Eventos}

Baseado nos eventos da metodologia ágil \textit{Scrum}, a equipe realiza algumas cerimônias, dentre elas: \textit{Stand-up}, Reuniões de planejamento (com \textit{Planning Poker}) e Reuniões de encerramento (com retrospectivas).


