\section{Metodologia}
\label{sec:metodologia}

\subsection{Escolha da equipe}
\label{sec:equipe}

A equipe é formada, majoritariamente, por alunos de graduação do curso de Engenharia de Software da Universidade de Brasília, conta ainda com dois ex-alunos formados, alunos de mestrado, que trabalham exclusivamente com o design, e professores orientadores. 
	
	Devido a esta formação, a equipe não consegue estar sempre trabalhando fisicamente junta, pois cada membro da equipe possui horários particulares de aula e, com isso, o horário dedicado a contribuição do projeto pelos mesmos depende do horário dessas aulas. Dessa forma, não conseguimos utilizar integralmente as práticas das metodologias ágeis.
	
	A equipe do projeto está dividida em duas equipes: equipe Noosfero e equipe Colab, a qual também atua na integração com outras ferramentas. A equipe do Noosfero está desenvolvendo um plugin com novas funcionalidades para o Portal do Software Público. A equipe do Colab está, no momento, trabalhando com configurações das ferramentas, pois está sendo realizada a integração do Redmine e do Gitlab com o Colab, o que exige maiores esforços de infra-estrutura.
	
	Para manter o controle das atividades do projeto e evitar os ruídos de comunicação temos uma lista de e-mail com todos os integrantes do projeto, onde foi criado o hábito de enviarmos um e-mail ao final do dia com as atividades que desenvolvemos, e a ferramenta de gerenciamento Redmine, onde são escritas as histórias de usuários, as histórias técnicas e suas respectivas tarefas. Realizamos ainda um stand-up em todos os dias de trabalho,em que todos estão ou no laboratório ou presente virtualmente para alinharmos a situação das equipes. 
	
	Estamos trabalhando em sprints/ciclos de duas semanas, em média, e releases de quatro meses, sendo que o projeto tem duração programada de três anos, sendo ao total sete releases e cinquenta e oito sprints.


\subsection{Métodos de desenvolvimento e Gestão do Projeto}
\label{sec:metodo-gestao}

A Engenharia de Software tem evoluído suas práticas e metodologias em busca de padrões que regem o desenvolvimento de software de qualidade dentro dos escopos, custos e prazos desejados. 
%
Dada a oportunidade de adoção de métodos ágeis no desenvolvimento do presente trabalho, a 
metodologia utilizada é baseada em uma combinação das metodologias Scrum e Extreme Programming. Destacam que XP e Scrum complementam um ao outro bem, com o XP provendo suporte para aspectos mais técnicos enquanto o Scrum provê práticas e técnicas para gerenciamento, planejamento e acompanhamento. Assim,com base nas experiências destacadas em~\cite{schwaber2001} e~\cite{fitzgerald2006}, na motivação de adoção de métodos ágeis no desenvolvimento de software moderno, serão apresentadas os métodos XP e Scrum, suas principais características e práticas que serão utilizadas no desenvolvimento do presente projeto.

%TODO ...

