\begin{abstract}
O Portal do Software Público Brasileiro (SPB), na prática, é um sistema web
que se consolidou como um ambiente de compartilhamento
de softwares. O projeto de evolução deste portal está sendo desenvolvido pela 
Universidade de Brasília e conta com integrantes de diversos perfis e níveis de formação. O 
projeto utiliza algumas práticas de métodos ágeis em sua metodologia de desenvolvimento. O 
objetivo deste artigo é relatar a experiência no desenvolvimento, ao lado do Ministério de Planejamento, bem como mostrar o quanto o projeto tem contribuído para a formação dos seus 
integrantes.

\end{abstract}
\vspace{1\baselineskip}

\begin{keywords}
Engenharia de Sofware, Software Livre, Software Público, Evolução de Software
\end{keywords}



