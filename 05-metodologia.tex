\section{Metodologia de Desenvolvimento}
\label{sec:metodologia}

A Engenharia de Software tem evoluído suas práticas e metodologias em busca de padrões que regem o desenvolvimento de software de qualidade dentro dos escopos, custos e prazos desejados. O modelo dito tradicional tem como característica um conjunto grande e detalhado de documentação que deve, supostamente, ser utilizada ao longo de todo o ciclo de desenvolvimento. Entretanto, percebendo que os objetivos principais do desenvolvimento de software não estavam sendo alcançados, alguns líderes da indústria e academia começaram a adotar métodos mais simples de trabalho que apresentaram melhores resultados em projetos de software. Em 2001, líderes que estavam desenvolvendo projetos fora dos padrões industriais se reuniram para trocar experiências e trabalhos. Este grupo se tornou a Aliança de Desenvolvimento Ágil e escreveram o Manisfesto Ágil que apresenta os princípios e valores que o grupo considera ser determinante para o desenvolvimento de software. Neste sentido, os métodos de desenvolvimento ágeis de software são métodos que implementam os seguintes valores:

\begin{itemize}

\item Indivíduos e interações acima de processos e ferramentas; 

\item Software operante acima de documentações grandes e completas; 

\item Colaboração do cliente acima de negociações contratuais; 

\item Responder à mudanças acima de seguir a um planejamento.

\end{itemize}

Esse conjunto de valores não descartam a importância dos elementos citados à direita das sen- 
tenças, mas evidenciam que estes são menos importantes diante dos primeiros elementos citados. 
Em outras palavras, apesar da documentação ser importante, o foco principal deve estar na entrega 
de software de valor para o cliente e na interação e a consequente comunicação entre as pessoas. 
Além disso, os métodos ágeis exaltam a simplicidade, feedback contínuo e adaptação à mudanças 
que podem ser obitidos a partir de comunicação face à face, qualidade de código e entrega contínua 
de software. 

Dada a oportunidade de adoção de métodos ágeis no desenvolvimento do presente trabalho, a 
metodologia utilizada é baseada em uma combinação das metodologias Scrum e Extreme Programming. Destacam que XP e Scrum complementam um ao outro bem, com o XP provendo suporte para aspectos mais técnicos enquanto o Scrum provê práticas e técnicas para gerenciamento, planejamento e acompanhamento. Assim,com base nas experiências destacadas em~\cite{schwaber2001} e~\cite{fitzgerald2006}, na motivação de adoção de métodos ágeis no desenvolvimento de software moderno, serão apresentadas os métodos XP e Scrum, suas principais características e práticas que serão utilizadas no desenvolvimento do presente projeto.
