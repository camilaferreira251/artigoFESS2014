\section{Conclusão}
\label{sec:conclusao}

%A partir do relato de experiência podemos concluir que os integrantes do projeto não tinham muita experiência nas ferramentas a serem utilizadas no projeto, e ainda assim os entrevistados conseguiram contribuir com o projeto.

%Foi possível concluir também que o projeto está contribuindo com a construção da formação dos alunos e ainda ajuda no desempenho dos alunos nas disciplinas da graduação devido ao conhecimento prévio adquirido. Mesmo que o tempo gasto trabalhando no projeto atrapalhe moderadamente o desempenho na graduação dos integrantes do projeto, o esforço será recompensado com uma boa experiência para o mercado de trabalho.

%De maneira geral o projeto tem ajudado os seus integrantes a aprender a lidar com dificuldades encontradas no desenvolvimento de software, está trazendo conhecimento técnico e gerencial para os membros da equipe além de auxiliar na relações interpessoais.

Concluímos que a utilização de metodologias ágeis contribuíram favoravelmente para o desenvolvimento do projeto, pois o escopo do mesmo está em contínua mudança e tais metodologias facilitaram a adaptação da equipe. Destacamos que XP e Scrum complementam um ao outro bem, com o XP provendo suporte para aspectos mais técnicos enquanto o Scrum provê práticas e técnicas para gerenciamento, planejamento e acompanhamento.