\section{Conclusão}
\label{sec:conclusao}

Neste projeto são utilizadas diversas ferramentas que não têm sua linguagem de programação em comum e sobretudo são relativamente novas no mercado e mesmo com esses impecilhos podemos afirmar, a partir do relato de experiência, que os alunos não tinham experiêcia nas ferramentas a serem utilizadas no projeto mas esta aparente dificuldade não impediu os alunos de contribuir com o projeto

Foi possível concluir também que o projeto está contribuindo com a construção da formação dos alunos, pois coloca em prática algumas situações antes vistas apenas na teoria dentro da sala de aula. 
%
O desempenho dos alunos nas disciplinas da graduação também é ajudado pelo projeto,devido ao conhecimento prévio adquirido não somente de práticas ágeis mas também de novas linguagens de programação. 
%
Ficou conhecido também que tempo gasto trabalhando no projeto atrapalha moderadamente o desempenho na graduação dos integrantes do projeto, mas o esforço será recompensado com uma boa experiência para o mercado de trabalho.

De maneira geral o projeto tem ajudado os seus integrantes a aprender a lidar com dificuldades encontradas no desenvolvimento de software, está trazendo conhecimento técnico e gerencial para os membros da equipe além de auxiliar nas relações interpessoais.